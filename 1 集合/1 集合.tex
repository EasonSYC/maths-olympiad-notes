%!TEX TX-program = xelatex
\documentclass[8pt]{article}

\usepackage[UTF8]{ctex}
\usepackage{graphicx}
\usepackage{enumerate}
\usepackage{geometry}
\usepackage{amsmath}
\usepackage{amssymb}
\usepackage{amsfonts}

\title{\S 1 集合}
\author{高一(6)班\ 邵亦成\ 26号}
\date{2021年09月11日}

\geometry{a4paper, scale=0.85}

\begin{document}

	\maketitle

	\begin{enumerate}
		\item

			已知:$S_1, S_2, S_3$ 为非空整数集,$\forall 1, 2, 3$的排列$i, j, k : x \in S_i, y \in S_j \Rightarrow x-y \in S_k$.

			求证:$S_1, S_2, S_3$ 三个集合中至少有两个相等.

			~\\
			思路:

			\begin{enumerate}[ (i) ]
				\item
					至少有两个相等 $\Leftrightarrow 0\in S_1 \cup S_2 \cup S_3$

					\begin{table}[h]
						\centering
						\begin{tabular}{|c|c|c|}
							\hline $S_1$&$S_2$&$S_3$\\
							\hline $0$&$x$&$x$\\
							\hline &$y$&$y$\\
							\hline &$\vdots$&$\vdots$\\
							\hline
						\end{tabular}
						\caption{若 $0 \in S_1 \cup S_2 \cup S_3$}
					\end{table}

				\item

					若$x\in S_i$则$-x \in S_i$

					\begin{table}[h]
						\centering
						\begin{tabular}{|c|c|c|}
							\hline $S_1$&$S_2$&$S_3$\\
							\hline $x$&$y$&$y-x$\\
							\hline $-x$&&\\
							\hline
						\end{tabular}
						\caption{若$x \in S_i$}
					\end{table}

				\item

					对于$y \in \mathbb{N}^*$
					\begin{enumerate}[ i) ]
						\item
							若$x>y$则$x$除以$y$的余数$\in S$
						\item
							若$0<x<y$则$x \in S$
					\end{enumerate}

					$\therefore \forall y \in \mathbb{N}^*:\exists x \in \mathbb{N} \cap S : x < y, y \in S$

					\begin{table}[h]
						\centering
						\begin{tabular}{|c|c|c|}
							\hline $S_1$&$S_2$&$S_3$\\
							\hline $x$&$y$&$x-y$\\
							\hline $x-2y$&&$x-3y$\\
							\hline $x-4y$&&$x-5y$\\
							\hline $\vdots$&&$\vdots$\\
							\hline
						\end{tabular}
						\caption{对于$y \in N^*$}
					\end{table}
			\end{enumerate}

			证明:

			\begin{enumerate} [ i)]

				\item
					若$0\in S_1 \cup S_2 \cup S_3$, 得证.

				\item
					若$0\notin S_1 \cup S_2 \cup S_3$, 设 $t=\min\left(S_1 \cup S_2 \cup S_3\right)$.

					不妨设 $t \in S_1$, 对于 $x \in S_2 \left(x > 0\right)$:

					$x-t \in S_3, x-2t \in S_2 \dots$, 即$x - \left[ \frac{x}{t} \right] t \in S_1 \cup S_3$.

					而 $0 \leq x - \left[ \frac{x}{t} \right] t < t$.

					结合$t$的最小性可知 $x - \left[ \frac{x}{t} \right] t =0 \in S_1 \cup S_3$

					与假设不符.

			\end{enumerate}
			
			综上所述, $S_1, S_2, S_3$ 三个集合中至少有两个相等.

		~\\

		\item

			设$a_1, a_2, a_3, a_4$是$4$个有理数,使:

			$$\left \{a_i a_j | 1 \leq i < j \leq 4, i,j \in \mathbb{Z} \right\} = \left\{-24, -2, -\frac{3}{2}, -\frac{1}{8}, 1, 3\right\}$$

			求$a_1 + a_2 + a_3 + a_4$.

			~\\
			$\prod ( a_i a_j ) = (-24) \times (-2) \times \left(-\frac{3}{2}\right) \times \left(-\frac{1}{8}\right) \times 1 \times 3 = 3^3$.

			$\therefore \prod a_i = 3$

			不妨设$a_1 a_2 = -24$, 则$a_3 a_4 = -\frac{1}{8}$.

			不妨设$a_1 a_3 = -2$, 则$a_2 a_4 = -\frac{3}{2}$.

			\begin{enumerate}[ (i) ]

				\item
					若$a_1 a_4 = 1, a_2 a_3 = 3$

					解得

					$$\left \{
						\begin{aligned}
						&a_1 = 4 \\
						&a_2 = -6 \\
						&a_3 = -\frac{1}{2}\\
						&a_4 = \frac{1}{4}
						\end{aligned}
					\right.$$

					或

					$$\left \{
						\begin{aligned}
						&a_1 = -4 \\
						&a_2 = 6 \\
						&a_3 = \frac{1}{2}\\
						&a_4 = -\frac{1}{4}
						\end{aligned}
					\right.$$

				\item
					若$a_1 a_4 = 3, a_2 a_3 = 1$, 无有理数解.

			\end{enumerate}

			综上所述,$a_1 + a_2 + a_3 + a_4 = \pm \frac{9}{4}$.

		~\\

		\item

			称有限集$S$的所有元素的积为$S$的“积数”. 给定 $M = \left \{ \frac{1}{2}, \frac{1}{3}, \cdots \frac{1}{100} \right \}$. 求$M$的所有偶数元子集的“积数”之和.

			~\\
			构造方程

			$$\left(x-\frac{1}{2}\right)\left(x-\frac{1}{3}\right)\cdots\left(x-\frac{1}{100}\right)=a_{99} x^{99} + a_{98} x^{98} + \cdots a_1 x^1 + a_0$$

			则

			$$\sum (x_i x_j) = \frac{a_{97}}{a_{99}}$$

			$$\sum (x_i x_j x_k x_m) = \frac{a_{95}}{a_{99}}$$

			$$\vdots$$

			于是对所求和$S$有:

			$$S=\frac{a_{97}+a_{95}+\cdots+a_1}{a_{99}}=\sum_{k=1}^{44} a_{2k+1}$$

			在所构造方程中代入$x=1$有:

			$$\frac{1}{100} = a_{99} + a_{97} + a_{95} + \cdots + a_1 + a_0$$

			代入$x=-1$有:

			$$-\frac{101}{2} = - a_{99} + a_{97} - a_{95} + \cdots - a_1 + a_0$$

			相减得

			$$\frac{5051}{100} = 2 \times \left ( 1 + \sum_{k=1}^{44} a_{2k+1} \right )$$

			$\therefore a_{97} + \cdots + a_1 = \frac{4851}{200}$.

		~\\

		\item

			设$n\in\mathbb{N}$, $S={1, \cdots, 2020}$. 求最小的$k \in \mathbb{N}^*$, 使得$\exists S$的子集$A_1, A_2, \cdots A_k$具有:对于$S$中的任意两个不同的元素$a, b$存在$j\in {1, 2, \cdots k}$, 使得 $A_j \cap \{a,b\}$的元素个数为1.

			~\\
			存在性问题只需构造一个符合条件的即可.

			考虑

			$$a \in \{1, 2, \cdots 1010\}, b \in \{1011, 1012, \cdots, 2020\}$$
			
			构造:

			\begin{table}[h]
				\centering
				\begin{tabular}{|c|c|c|c|c|c|c|c|c|}
					\hline &$1$&$2$&$3$&$\cdots$&$1010$&$1011$&$\cdots$&$2020$\\
					\hline $A$&$\surd$&$\surd$&$\surd$&$\cdots$&$\surd$&$\times$&$\cdots$&$\times$\\
					\hline
				\end{tabular}
				\caption{考虑一种$A$}
			\end{table}
			~\\
			余下的情形:

			$$a,b \in \{1, 2, \cdots 1010\}$$

			或

			$$a,b \in \{1011, 1012, \cdots, 2020\}.$$

			若$i\in A_j$, 我们在$i$所在列, $A_j$所在行填入$1$, 反之填入$0$.

			将$i$所在列构成的序列记为$P(i)$,

			由题可知, $\forall a, b \in S, a \neq b : P(a) \neq P(b)$.

			$2^{10} < 2020 < 2^{11}$, $\therefore k\geq 11$.

		~\\

		\item

			设$S=\left \{ A_1, A_2, \cdots A_n \right \}$, 其中$A_1, A_2, \cdots a_n$是$n$个互不相同的有限集$(n \geq 2)$满足$\forall A_i, A_j \in S : A_i \cup A_j \in S, k = \min_{1<j\leq n}\left|A_i\right|\geq 2$.

			求证:

			$$\exists x \in \bigcup_{i=1}^{n} A_i : x \in A_1 \cdots A_n$$

			满足$x \in A_1 \cdots A_n$的至少$n$个集合.

			不妨设

			$$\left| A_1 \right| \leq \left| A_2 \right| \leq \cdots \left| A_n \right|$$

			记$A_1=\left \{ x_1, x_2, \cdots x_k \right \}$,

			\begin{enumerate} [ (i) ]
				\item
					$A_i \subseteq A_n (i = 1, 2, \cdots n)$
					因为$A_n \subseteq A_i \cup A_n \in S$ 
				
				\item
					记$P{x_i)=\{A_j|x_i \in A_j\, i = 1 \cdots k, j = 1 \cdots n}$
					只需证
					$$\exists i: \left|P(x_i)\right| \geq \frac{n}{k}$$
					只需证
					$$\sum_{i=1}^k \left|P(x_i)\right| \geq n$$
					只需证
					$$\forall j:\exists i: A_j \in P(x_2) \Leftrightarrow A_j, A_i \cap A_j \neq \emptyset$$
					但此式不成立.
			\end{enumerate}

			记$A_{i1}, A_{i2}, \cdots A_{ik}$是$S$中与$A_1$不相交的集合.

			则有$A_{i1}\cup A_1 \in S, A_{i2}\cup A_1 \in S, \cdots A_{ik}\cup A_1 \in S$

			\begin{table}[h]
				\centering
				\begin{tabular}{|c|c|c|c|c|c|c|c|c|c|c|c|c|c|}
					\hline &$A_1$&$A_{i1}$&$A_{i2}$&$\cdots$&$A_{it}$&$A_{i1}\cup A_1$&$A_{i2}\cup A_2$&$\cdots$&$A_{it}\cup A_1$&$A_1^{'}$&$A_2^{'}$&$\cdots$&$A_{n-2t-1}^{'}$\\
					\hline $x_1$&1&0&0&$\cdots$&0&1&1&$\cdots$&1&&&&\\
					\hline $x_2$&1&0&0&$\cdots$&0&1&1&$\cdots$&1&&&&\\
					\hline $\vdots$&\vdots&\vdots&\vdots&$\cdots$&\vdots&\vdots&\vdots&$\cdots$&1&&&&\\
					\hline $x_k$&1&0&0&$\cdots$&0&1&1&$\cdots$&1&&&&\\
					\hline
				\end{tabular}
				\caption{构造集合$A$}
			\end{table}

			$\sum_{i=1}^{k} |P(x_1)|$即为表格中所有数的和.

			令$N$为表格中剩余$n-2t-1$列所有数字的和,

			$$\begin{aligned}
				\sum_{i=1}^{k} |P(x_1)| &= k + kt + N\\
				&\geq k + kt + (n - 2t - 1)\\
				&\geq 2 + 2t + n - 2t - 1\\
				&= n + 1\\
				&> n\\
			\end{aligned} $$

			得证.

	\end{enumerate}
\end{document}