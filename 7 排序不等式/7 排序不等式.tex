%!TEX TX-program = xelatex
\documentclass[8pt]{article}

\usepackage{ctex}
\usepackage{graphicx}
\usepackage{enumitem}
\usepackage{geometry}
\usepackage{amsmath}
\usepackage{amssymb}
\usepackage{amsfonts}
\usepackage{tikz}
\usepackage{extarrows}
\usetikzlibrary{positioning}
\usepackage{xcolor}

\graphicspath{ {./images/} }

\title{\S 7 排序不等式}
\author{高一(6)班\ 邵亦成\ 26号}
\date{2021年11月13日}

\geometry{a4paper, scale=0.85}

\begin{document}

	\maketitle

	\textbf{排序不等式 (Rearrangement Inequality)}: 设$a_1 \leq a_2 \leq \cdots \leq a_n, b_1 \leq b_2 \leq \cdots \leq b_n, j_1, j_2, \cdots, j_n$是$1, 2, \cdots, n$得任意排列, 则$a_1 b_n + a_2 b_{n-1} + \cdots + a_{n-1} b_2 + a_n b_1 \leq a_1 b_{j_1} + a_2 b_{j_2} + \cdots + a_n b_{j_n} \leq a_1 b_1 + a_2 b_2 + \cdots + a_n b_n$, 等号成立当且仅当$a_1 = a_2 = \cdots = a_n$或$b_1 = b_2 = \cdots = b_n$.

	\textbf{切比雪夫 (总和) 不等式 (Chebyshev's Sum Inequality)}: 设$a_1 \leq a_2 \leq \cdots \leq a_n, b_1 \leq b_2 \leq \cdots \leq b_n, $则$n(a_1 b_n + a_2 b_{n-1} + \cdots + a_{n-1} b_2 + a_n b_1) \leq (a_1 + a_2 + \cdots + a_{n-1} + a_n)(b_1 + b_2 + \cdots + b_{n-1} + b_n) \leq n(a_1 b_1 + a_2 b_2 + \cdots + a_n b_n)$, 等号成立当且仅当$a_1 = a_2 = \cdots = a_n$或$b_1 = b_2 = \cdots = b_n$.

	~\\

	\begin{enumerate}
		\item 证明排序不等式.
			~\\

			记$S(k_1, k_2, \cdots, k_n)=\sum_{i=1}^{n} a_i b_{k_i}$,

			若$k_n \neq n$, 设$k_p = n$,

			\begin{align*}
				&S(k_1, k_2, \cdots, k_n) - S(k_1, k_2, \cdots, k_n, k_{p+1}, \cdots, k_{n-1}, k_p)\\
				=& a_p b_{k_p} + a_n b_{k_n} - a_p b_{k_n} - a_n b_{k_n}\\
				=& (a_p - a_n) (b_{k_p} - b_{k_n})\\
				\leq& 0.\\
			\end{align*}

			当$S(k_1, k_2, \cdots, k_n)$最大时, $k_n = n$. 同理$k_{n-1} =  n-1, \cdots, k_2 = 2, k_1 = 1.$

		~\\
	
		\item 证明切比雪夫不等式.
			~\\

			\begin{align*}
				(a_1 + a_2 + \cdots + a_n)(b_1 + b_2 + \cdots + b_n) =& (a_1 b_1 + a_2 b_2 + \cdots + a_n b_n)\\
				& + (a_1 b_2 + a_2 b_3 + \cdots + a_n b_1)\\
				& + \cdots\\
				& + (a_1 b_n + a_2 b_1 + \cdots + a_n b_{n-1}).
			\end{align*}

		~\\

		\item 设$a, b, c, \lambda \in \mathbb{R}^{+}$ 且 $a^{n-1}+b^{n-1}+c^{n-1} = 1 (n\geq 2)$. 求证: $$\frac{a^n}{b+\lambda c} + \frac{b^n}{c+\lambda a} + \frac{c^n}{a+\lambda b}\geq \frac{1}{1+\lambda}.$$
			~\\

			$$\left(\frac{a^n}{b+\lambda c}+\frac{b^n}{c+\lambda a}+\frac{c^n}{a+\lambda b}\right)\left[a^{n-2}(b+\lambda c)+b^{n-2}(c+\lambda a)+c^{n-2}(a+\lambda b)\right]\geq\left(a^{n-1}+b^{n-1}+c^{n-1}\right)^2=1.$$

			只需证

			$$a^{n-2}(b+\lambda c)+b^{n-2}(c+\lambda a)+c^{n-2}(a+\lambda b)\leq 1+\lambda.$$

			\begin{align*}
				a^{n-2}b+b^{n-2}c+c^{n-2}a &\leq a^{n-2} a+b^{n-2} b+c^{n-2} c + \lambda{a^{n-2} c+b^{n-2} a+c^{n-2} b}\\
				&\leq (1+\lambda) \left(a^{n-2}a+b^{n-2}b+c^{n-2}c\right)\\
				&= (1+\lambda) \left(a^{n-1} b^{n-1} c^{n-1}\right)\\
				&= 1+\lambda.
			\end{align*}

		~\\

		\item 设$x_1, x_2, \cdots, x_n$与$a_1, a_2, \cdots, a_n (n>2)$满足条件: (1) $x_1 + x_2 + \cdots + x_n = 0$, (2) $|x_1|+|x_2|+\cdots+|x_n|=1$, (3) $a_1 \geq a_2 \geq \cdots \geq a_n$的两组任意实数. 为了使不等式$|a_1 x_1 + a_2 x_2 + \cdots + a_n x_n| \leq A(a_1 - a_n)$成立, 求数$A$的最小值.
			~\\

			显然

			$$A_{\min}=\left(\frac{a_1 x_1 + a_2 x_2 + \cdots + a_n x_n}{a_1 - a_n}\right)_{\max}.$$

			不妨设$a_1 x_1 + a_2 x_2 + \cdots + a_n x_n \geq 0$. (若$<0$, 将$x_i$与$-x_i$替换), 设$x_{k_1} \geq x_{k_2} \geq \cdots \geq x_{k_t} \geq 0 > x_{k_{t+1}} \geq \cdots \geq x_{k_n}$, 则有

			$$\sum_{i=1}^{n} x_{k_i}=\frac{1}{2}, \sum_{i=k+1}^{n} x_{k_i}=-\frac{1}{2},$$

			\begin{align*}
				a_1 x_1 + a_2 x_2 + \cdots + a_n x_n &\leq a_1 x_{k_1} + a_2 x_{k_2} + \cdots + a_n x_{k_n}\\
				&= a_1 (x_{k_1} + x_{k_2} + \cdots + x_{k_t}) + a_n (x_{k_{t+1}} + \cdots + x_{k_n})\\
				&= \frac{1}{2} (a_1 - a_n)\\
				\Rightarrow& A_{\min} = \frac{1}{2},
			\end{align*}

			等号成立条件: $\displaystyle x_1=\frac{1}{2}, x_n=-\frac{1}{2}, x_2 = \cdots = x_{n-1} = 0.$

		~\\

		\item 设$x, y, z \in \mathbb{R}^{+}$, 求证: $$\frac{x}{\sqrt{y+z}}+\frac{y}{\sqrt{z+x}}+\frac{z}{\sqrt{x+y}}\geq\sqrt{\frac{3}{2}(x+y+z)}$$.
			~\\

			不妨设$x\geq y\geq z,$ 则有

			$$\frac{1}{\sqrt{y+z}}\geq \frac{1}{\sqrt{z+x}}\geq \frac{1}{\sqrt{x+y}},$$

			有

			$$\text{左} \geq \left[\frac{1}{3} (x+y+z)\left(\frac{1}{\sqrt{y+z}}+\frac{1}{\sqrt{z+x}}+\frac{1}{\sqrt{x+y}}\right)\right]^2.$$

			只需证

			$$\left[\frac{1}{3} (x+y+z)\left(\frac{1}{\sqrt{y+z}}+\frac{1}{\sqrt{z+x}}+\frac{1}{\sqrt{x+y}}\right)\right]^2 \geq \frac{3}{2}.$$

			设$y+z=a, z+x=b, x+y=c$,

			\begin{align*}
				\text{左}&= \frac{1}{9} \cdot \frac{a+b+c}{2} \left(\frac{1}{\sqrt{a}}+\frac{1}{\sqrt{b}}+\frac{1}{\sqrt{c}}\right)^2\\
				&\geq \frac{1}{2} (1+1+1)^3\\
				&=\frac{27}{2}. \tag*{\text{卡尔松不等式}}\\
			\end{align*}

		~\\

		\item 设$a\leq b\leq c\leq d\leq e$, 且$a+b+c+d+e=1$. 求证: $$ad+dc+cb+be+ea\leq \frac{1}{5}.$$
			~\\

			$$2(ad+dc+cb+be+ea)=a(d+e)+b(c+e)+c(b+d)+d(a+c)+e(a+b).$$

			$$a\leq b\leq c\leq d\leq e \Rightarrow d+e\geq c+e\geq b+d\geq a+c\geq a+b,$$

			由切比雪夫不等式,

			$$a(d+e)+b(c+e)+c(b+d)+d(a+c)+e(a+b)\leq \frac{(a+b+c+d+e)[d+e+a+e+b+d+a+c+a+b]}{5} =\frac{2}{5}.$$

		~\\

		\item $x_1, x_2, \cdots x_n > 0$, $\displaystyle \sum_{i=1}^{n} x_i = 1 (n \geq 2).$ 求证: $$\sum_{i=1}^{n} \frac{1}{1-x_i}\cdot\sum_{1\leq i<j\leq n} x_i x_j\leq \frac{n}{2}.$$
			~\\

			\begin{align*}
				\text{左} &= \sum_{i=1}^{n} \frac{1-x_i} \cdot \frac{1}{2} \sum_{i=1}^{n} x_i (x_1 + x_2 + \cdots + x_{i-1} + x_{i+1} + \cdots + x_n)\\
				&= \frac{1}{2} \sum_{i=1}^{n} \frac{1}{1-x_i} \cdot \sum_{i=1}^{n} x_i (1-x_i).
			\end{align*}

			不妨设$x_1\geq \cdots \geq x_n$, 则有$\displaystyle \frac{1}{1-x_1} \geq \cdots \geq \frac{1}{1-x_n}.$

			由切比雪夫不等式,

			$$\sum_{i=1}^{n} \frac{1}{1-x_i} \cdot \sum_{i=1}^{n} x_i (1-x_i) \leq n \sum_{i=1}^{n} \frac{1}{1-x_i} \cdot x_i (1-x_i)=n.$$

		~\\

		\item 已知实数$x_1 < x_2 < \cdots < x_n$, $y_1 < y_2 < \cdots < y_n$ 满足$\displaystyle \sum_{i=1}^{n} x_i = \sum_{i=1}^{n} y_i = 0$. 求证: (1) $\displaystyle \sum_{i=1}^{n} x_i y_i \geq \frac{n}{n-1} \max_{1\leq i\leq n} x_i y_i$, (2) $\displaystyle (n-1)^2 \left(\sum_{i=1}^{n} x_i y_i\right)^2 \geq \left(\sum_{i=1}^{n} x_i^2\right)^2 \left(\sum_{i=1}^{n} y_i^2\right)^2.$
			~\\

			\textbf{(1)} 

			\begin{align*}
				x_1 y_1 &= (x_2 + \cdots + x_n)(y_2 + \cdots + y_n)\\
				&\leq (n-1)(x_2 y_2 + x_3 y_3 + \cdots + x_n y_n)\\
				&= (n-1) \left(\sum_{i=1}^{n} x_i y_i - x_1 y_1 \right).
			\end{align*}

			同理, $n(x_n y_n) \leq (n-1) \sum_{i=1}^{n} x_i y_i.$

			\textbf{(2)}

			$$(n-1)^2 \left(\sum_{i=1}^{n} x_i y_i \right)^2 \geq (n x_1 y_1)(n x_n y_n).$$
			
			只需证

			$$-n x_1 x_n \geq \sum_{i=1}^{n} x_i^2.$$

			($y$同理.)

			$$(x_i - x_1)(x_i - x_n) \leq 0 \Rightarrow x_i^2 - (x_1 + x_n) x_i + x_1 x_n \leq 0.$$

			求和, 有:

			$$\sum_{i=1}^{n} x_i^2 - (x_1 + x_n) \sum_{i=1}^{n} x_i + n x_1 x_n\leq 0,$$

			即有:

			$$\sum_{i=1}^{n} x_i^2 \leq -n x_1 x_n.$$

		~\\

		\item 设$x_1, x_2, \cdots, x_n > 0$, 求证: $$\frac{1}{\frac{1}{1+x_1}+\frac{1}{1+x_2}+\cdots+\frac{1}{1+x_n}} - \frac{1}{\frac{1}{x_1}+\frac{1}{x_2}+\cdots+\frac{1}{x_n}}\geq\frac{1}{n}.$$
			~\\

			即证

			$$\frac{\frac{1}{x_1}+\frac{1}{x_2}+\cdots+\frac{1}{x_n} - \frac{1}{1+x_1}-\frac{1}{1+x_2}-\cdots-\frac{1}{1+x_n}}{\left(\frac{1}{1+x_1}+\frac{1}{1+x_2}+\cdots+\frac{1}{1+x_n}\right)\left(\frac{1}{x_1}+\frac{1}{x_2}+\cdots+\frac{1}{x_n}\right)} \geq \frac{1}{n},$$

			不妨设$x_1 \geq x_2 \geq \cdots \geq x_n$, 则有$\displaystyle \frac{1}{1+x_1} \leq \frac{1}{1+x_2} \leq \cdots \leq \frac{1}{1+x_n}.$

			由切比雪夫不等式,

			$$\frac{\sum_{i=1}^{n} \frac{1}{x_i} - \sum_{i=1}^{n}\frac{1}{1+x_i}}{\sum_{i=1}^{n} \frac{1}{x_i} \cdot \sum_{i=1}^{n}\frac{1}{1+x_i}} \leq n\sum_{i=1}^{n} \frac{1}{x_i} \frac{1}{x_i + 1} = n\left(\sum_{i=1}^{n}\frac{1}{x_i} - \sum_{i=1}^{n}\frac{1}{1+x_i}\right),$$

			即

			$$\frac{\frac{1}{x_1}+\frac{1}{x_2}+\cdots+\frac{1}{x_n} - \frac{1}{1+x_1}-\frac{1}{1+x_2}-\cdots-\frac{1}{1+x_n}}{\left(\frac{1}{1+x_1}+\frac{1}{1+x_2}+\cdots+\frac{1}{1+x_n}\right)\left(\frac{1}{x_1}+\frac{1}{x_2}+\cdots+\frac{1}{x_n}\right)} \geq \frac{1}{n}.$$

		~\\

		\item $n\geq 2, 0 \leq x_1 \leq x_2 \leq \cdots \leq x_n, \displaystyle x_1 \geq \frac{x_2}{2} \geq \cdots \geq \frac{x_n}{n},$ 证明: $$\frac{\sum\limits_{i=1}^{n} x_i}{n\sqrt[n]{\prod\limits_{i=1}^{n}x_i}}\leq\frac{n+1}{2\sqrt[n]{n!}}.$$
			~\\

			$$\frac{1}{x_1}+\frac{2}{x_2}+\cdots+\frac{n}{x_n} \geq n\sqrt[n]{\frac{n!}{\prod\limits_{i=1}^{n} x_i}},$$

			$$\text{原不等式}\Leftrightarrow n\sqrt[n]{\frac{n!}{\prod\limits_{i=1}^{n} x_i}} \leq \frac{(n+1)n^2}{2\sum\limits_{i=1}^{n} x_i},$$

			只需证

			$$\sum_{i=1}^{n} \frac{i}{x_i} \leq \frac{(n+1)n^2}{2\sum\limits_{i=1}^{n} x_i}.$$

			由切比雪夫不等式,

			$$\sum_{i=1}^{n} \frac{i}{x_i} \cdot \sum_{i=1}^{n} x_i \leq n\sum_{i=1}^{n} i = \frac{n^2(n+1)}{2}.$$

		~\\

		\item 设$x_1, x_2, \cdots, x_n > 0, k\geq 1$, 求证: $$\sum_{i=1}^{n} \frac{1}{1+x_i} \sum_{i=1}^{n} x_i \leq \sum_{i=1}^{n} \frac{x_i^{k+1}}{1+x_i} \sum_{i=1}^{n} \frac{1}{x_i^k}.$$
			~\\

			不妨设$x_1 \geq \cdots \geq x_n$, 有

			$$\frac{1}{x_1^k}\leq \frac{1}{x_2^k} \leq \cdots\leq\frac{1}{x_n^k},$$

			则有

			$$\frac{x_1^{k+1}}{1+x_1} \geq \frac{x_2^{k+1}}{1+x_2} \geq \cdots \geq \frac{x_n^{k=1}}{1+x_n}.$$

			由切比雪夫不等式, 有

			$$\sum x_i \frac{x_i^k}{1+x_i} \geq \frac{\left(\sum x_i\right)\left(\sum\frac{x_i^k}{1+x_i}\right)}{n},$$

			$$\text{右} \geq \left(\sum x_i\right)\frac{1}{n}\left(\sum \frac{x_i^k}{1+x_i}\right)\left(\sum \frac{1}{x_i^k}\right).$$

			由切比雪夫不等式, 有

			$$\frac{1}{n}\left(\sum \frac{x_i^k}{1+x_i}\right)\left(\sum \frac{1}{x_i^k}\right) \geq \sum\frac{1}{1+x_i}.$$

	\end{enumerate}
\end{document}